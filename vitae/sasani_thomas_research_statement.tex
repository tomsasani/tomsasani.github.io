%% start of file `template.tex'.
%% Copyright 2006-2010 Xavier Danaux (xdanaux@gmail.com).
%% Copyright 2010-2011 Mark Liu (markwayneliu@gmail.com).
%
% This work may be distributed and/or modified under the
% conditions of the LaTeX Project Public License version 1.3c,
% available at http://www.latex-project.org/lppl/.

\documentclass[11pt,a4paper,sans]{moderncv}

\usepackage{verbatim}

% moderncv themes
\moderncvstyle{classic}
\moderncvcolor{blue} 

% character encoding
\usepackage[utf8]{inputenc}                   % replace by the encoding you are using

% adjust the page margins
\usepackage[scale=0.8]{geometry}
%\setlength{\hintscolumnwidth}{3cm}						% if you want to change the width of the column with the dates
%\AtBeginDocument{\setlength{\maketitlenamewidth}{6cm}}  % only for the classic theme, if you want to change the width of your name placeholder (to leave more space for your address details
%\AtBeginDocument{\recomputelengths}                     % required when changes are made to page layout lengths


% personal data
\firstname{Thomas A.}
\familyname{Sasani}
\email{thomas.a.sasani@gmail.com}                      % optional, remove the line if not wanted
\homepage{http://tomsasani.github.io}   
\extrainfo{GitHub: \Colorhref{https://github.com/tomsasani}{@tomsasani}}             % optional, remove the line if not wanted
%\extrainfo{\url{http://markliu.me}} % optional, remove the line if not wanted

% to show numerical labels in the bibliography; only useful if you make citations in your resume
%\makeatletter
%\renewcommand*{\bibliographyitemlabel}{\@biblabel{\arabic{enumiv}}}
%\makeatother

%\nopagenumbers{}                             % uncomment to suppress automatic page numbering for CVs longer than one page
%----------------------------------------------------------------------------------
%            content
%----------------------------------------------------------------------------------
\begin{document}
\definecolor{color1}{rgb}{0.22,0.45,0.70}
\newcommand\Colorhref[3][color1]{\href{#2}{\small\color{#1}#3}}
%\maketitle

Given the enormous complexity of genome replication and repair pathways, and 
faced with myriad sources of potential DNA damage, our cells manage to transmit 
their genetic material with remarkably high fidelity. When the safeguards of 
genome integrity fail, germline and somatic mutations provide the raw 
material for both evolution and genetic disease. Throughout my career, I've been 
fascinated by the forces that contribute to the evolution of mutation rates in 
mammalian genomes. \textbf{Using new statistical and computational approaches,  my research 
program will investigate the genetic underpinnings of mutation rate 
and spectrum evolution in mammalian genomes.}

\section{Past and current work}

\textbf{Characterizing germline mutation rates and spectra in large 
human families:}  During my Ph.D., I used whole-genome sequencing data from large, three-generation 
pedigrees to precisely measure germline mutation rates and 
investigate the endogenous and exogenous factors that contribute to mutation rate 
variation in human families (Sasani et al., 2019, \emph{eLife}). I developed methods to confidently 
discriminate germline mutations from mutations that occurred after zygote 
fertilization, and discovered that over 10\% of apparent germline \emph{de novo} mutations 
were, in fact, post-zygotic in origin. I also found that children born to older 
parents possessed more germline mutations, and that these parental age effects 
varied more than ten-fold across families. At the time, I hypothesized that 
genetic modifiers of the mutation rate -- termed \emph{mutator alleles} -- might be 
responsible for these differences. 
\break \break
\textbf{Discovering germline mutator alleles in recombinant inbred mice:}  Compared 
to haplotypes that harbor wild-type alleles at a particular locus, 
those with mutator alleles are expected to carry an excess total number 
of \emph{de novo} germline mutations. Owing to the high fidelity of germline DNA 
replication and repair, which leads to a very low baseline mutation rate, as 
well as the strength of selection against a potential mutator, it can be 
difficult to test that expectation in mammalian genomes. As a postdoctoral 
fellow, I hypothesized that germline mutator alleles might be detectable by 
instead comparing the spectrum of mutation - the frequency of each individual 
mutation type (C$\rightarrow$T, A$\rightarrow$G, etc.) - between haplotypes. Since DNA replication and 
repair proteins often recognize particular sequence motifs or excise lesions at 
specific nucleotides, the mutation spectrum can be a rich source of information 
about genetic and environmental factors that affect mutagenesis. To test this 
hypothesis, I identified germline mutations in 152 recombinant inbred 
lines (RILs) called the BXDs, which were derived from two laboratory strains 
of mice (C57\underline{B}L/6J and \underline{D}BA/2J) that exhibited significant differences in their 
germline mutation spectra. Using quantitative trait locus mapping, I 
discovered a locus that was associated with a 50\% increase 
in the C$\rightarrow$A germline mutation rate (Sasani et al., 2022, \emph{Nature}). The locus 
overlapped \emph{Mutyh}, a protein-coding gene that normally prevents C$\rightarrow$A mutations by 
repairing a specific type of oxidative DNA damage. 
\break \break
\textbf{Identifying epistasis between mutator alleles:}  Empowered by the 
discovery of a large-effect germline mutator allele in mice, I became interested 
in developing new methods to detect additional mutation 
rate modifiers in model systems. As a staff scientist, I developed a new 
statistical method to detect alleles that affect the germline mutation spectrum 
in biparental RILs. By applying this method to the data I previously generated 
in the BXDs, I discovered an additional mutator locus that augments the C$\rightarrow$A 
germline mutation rate (Sasani et al., 2023, \emph{bioRxiv}). The locus overlaps a 
protein-coding gene called \emph{Ogg1}, which is a key partner of \emph{Mutyh} in the 
base-excision repair of oxidative DNA damage. Strikingly, BXDs with the mutator 
allele near \emph{Ogg1} did not exhibit elevated rates of C$\rightarrow$A germline mutation unless 
they also possessed the mutator allele near \emph{Mutyh}. However, BXDs with both 
alleles exhibited even higher C$\rightarrow$A mutation rates than those with either one 
alone, providing evidence of epistasis between germline mutator alleles for 
the first time in a mammalian system.

\section{Future research program}

\textbf{Aim 1: Discover novel drivers of cancer progression by analyzing somatic 
mutation spectra}

\underline{Cancer genomes are an ideal system for mutator allele discovery}

Although mutator alleles likely play a fundamental role in the evolution of 
germline mutation rates, they can also have a profound impact on health and disease, 
particularly in human cancers. "Driver mutations," which precipitate the transition
between a healthy and cancerous cell state, are often found in protein-coding genes
that contribute to DNA replication or repair pathways. Identifying these driver
mutations is critically important to early and accurate cancer diagnosis, and 
I hypothesize that new methods for discovering mutator alleles can be adapted 
for this purpose. Like the genomes of recombinant inbred lines, cancer genomes 
present a remarkably well-powered system for mutator allele detection. In 
sexually-reproducing populations of large effective population size, negative 
selection will efficiently remove large-effect mutator alleles. In cancer, 
however, the effects of negative selection are drastically attenuated, allowing 
mutators to persist and exert effects over the course of many cell divisions. 
In sexual species, recombination will also break up linkage between mutator 
alleles and the excess deleterious mutations they cause, making it difficult to
discover associations between the two; cancer genomes do not recombine, steadily
accumulating mutations that remain perfectly linked to a mutator. 
\break \break
\underline{Cancer mutation spectra are a rich source of information about the mutation process}

A number of methods have recently been developed to decompose cancer mutation data 
into "mutational signatures," which describe the relative frequencies of particular 
mutation types (e.g., TCC$\rightarrow$TTC, ATG$\rightarrow$AAG). 
In many cases, these signatures reveal the subtle activities of exogenous or 
endogenous mutation rate modifiers, and can even be used to accurately discriminate 
subtypes of a single cancer. However, mutation signatures describe global features 
of the mutation process, and cannot necessarily pinpoint causal genes or alleles that affect 
mutation rates or spectra. The Cancer Genome Atlas (TCGA) has collected hundreds of 
thousands of cancer biopsies from dozens of human tissues, and has made mutation 
data from these cancer cells available to the academic community. \textbf{By adapting 
methods for mutator allele discovery to somatic mutation data from human cancers, 
my group will search for novel drivers of cancer initiation and progression.}
\break \break
\textbf{Aim 2: Develop scalable approaches to discover active mutator alleles in diverse human genomes}
\underline{Germline mutator alleles likely shape mutation spectrum variation in human genomes}

Germline mutator allele discovery has largely been limited to model systems, 
including yeast, \emph{E. coli}, and mice. The degree to which mutator alleles shape 
germline mutation rates in human populations, however, remains unclear. Candidate 
germline mutator alleles have been discovered by identifying human 
haplotypes with excess counts of derived alleles, but their effects could not be 
replicated using de novo germline mutation data from other cohorts. The heritability
of paternal de novo mutation rates is also estimated to be between 10 and 20\%, 
and a number of recent studies have suggested that germline mutation rate 
modifiers have shaped the evolution of the mutation spectrum in humans and 
great apes. Simulations also demonstrate that given a sufficient number of 
moderate- to large-effect segregating germline mutator alleles, a fraction 
should be detectable using human trio data. 
\break \break
\underline{The utility of methods to detect mutation spectrum modifiers in human genomes}

The mutation spectrum provides a rich source of information about the demographic 
histories of populations, as well as the historical activities of exogenous and 
endogenous mutation rate modifiers. To this end, \textbf{my group will develop new methods
to detect mutator alleles using mutation spectra derived from human pedigree 
sequencing data and phased haplotype data from large cohorts of unrelated 
individuals.} Many research groups, including my own, have generated sequencing 
data from large cohorts of human families, enabling the identification of a 
large number of de novo germline mutations. Massive population sequencing efforts,
such as those undertaken by the TOPMED consortium, have also provided massive 
collections of derived singletons (mutations present on a single haplotype in 
a population) that can serve as rough proxies for recent de novo mutations. 
These efforts will involve a number of distinct projects, including simulations
to assess our power to detect alleles that affect mutation spectra, and 
refining my existing methods for mutator allele detection to operate on 
segregating genetic variation instead of de novo mutation counts.
\break 

\textbf{Aim 3: Improve mutation spectrum inference using pangenome graphs}

\underline{Pangenome graphs capture more genetic variation than linear reference assemblies}


As a representation of global genetic diversity, the linear human reference genome 
is a crude approximation. The most recent reference assembly is mostly derived from
the DNA of a single individual, and therefore captures a tiny sliver of the 
sequence variation that exists across the globe. Pangenome representations have 
recently emerged as an alternative to existing reference assemblies. A pangenome 
is a directed graph in which nodes correspond to nucleotide sequences and edges 
describe the connections between nodes that comprise a haplotype. As a result, 
pangenomes can be iteratively updated with new haplotype information; and by 
tracing new paths through a pangenome graph, we can represent far more of the 
genetic diversity in a population. By incorporating high-quality, haplotype-phased 
assemblies, the graph can also capture genetic diversity in 
structurally complex regions of the genome that are often absent from linear 
reference sequences. 
\break \break
\underline{Pangenome graphs can enable more precise and comprehensive analyses of mutation rates and spectra}


Pangenome representations offer a number of benefits for analyzing mutation rates 
and spectra, namely the ability to detect more mutations in more of the genome. 
Since the current linear reference assembly is largely derived from a single 
individual, the vast majority of human genetic variation is missing from the 
reference. As a result, sequencing reads that harbor such variation often map 
poorly to the linear reference assembly and are ignored when calling DNA variants;
thus, the homogeneity of the linear reference assembly can thwart our ability 
to discover new mutations in diverse, newly sequenced human genomes. Moreover, 
mutation rates and spectra exhibit significant spatial variation throughout the 
genome, and recent studies have suggested that mutation rates are elevated in 
segmental duplications and other structurally complex regions that are largely 
missing from the linear reference. By developing new methods for inferring mutation
spectra from pangenome representations, \textbf{my research group will more precisely 
characterize the human mutation spectrum across diverse populations, investigate
spatial variation in human mutation rates and spectra, and explore the mutation
process in structurally complex regions of the genome.}

\end{document}
