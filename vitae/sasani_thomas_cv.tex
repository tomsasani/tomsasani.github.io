%% start of file `template.tex'.
%% Copyright 2006-2010 Xavier Danaux (xdanaux@gmail.com).
%% Copyright 2010-2011 Mark Liu (markwayneliu@gmail.com).
%
% This work may be distributed and/or modified under the
% conditions of the LaTeX Project Public License version 1.3c,
% available at http://www.latex-project.org/lppl/.

\documentclass[11pt,a4paper,sans]{moderncv}

\usepackage{verbatim}

% moderncv themes
\moderncvstyle{classic}
\moderncvcolor{blue} 

% character encoding
\usepackage[utf8]{inputenc}                   % replace by the encoding you are using

% adjust the page margins
\usepackage[scale=0.8]{geometry}
%\setlength{\hintscolumnwidth}{3cm}						% if you want to change the width of the column with the dates
%\AtBeginDocument{\setlength{\maketitlenamewidth}{6cm}}  % only for the classic theme, if you want to change the width of your name placeholder (to leave more space for your address details
%\AtBeginDocument{\recomputelengths}                     % required when changes are made to page layout lengths


% personal data
\firstname{Thomas A.}
\familyname{Sasani}
\email{thomas.a.sasani@gmail.com}                      % optional, remove the line if not wanted
\homepage{http://tomsasani.github.io}   
\extrainfo{GitHub: \Colorhref{https://github.com/tomsasani}{@tomsasani}}             % optional, remove the line if not wanted
%\extrainfo{\url{http://markliu.me}} % optional, remove the line if not wanted

% to show numerical labels in the bibliography; only useful if you make citations in your resume
%\makeatletter
%\renewcommand*{\bibliographyitemlabel}{\@biblabel{\arabic{enumiv}}}
%\makeatother

%\nopagenumbers{}                             % uncomment to suppress automatic page numbering for CVs longer than one page
%----------------------------------------------------------------------------------
%            content
%----------------------------------------------------------------------------------
\begin{document}
\definecolor{color1}{rgb}{0.22,0.45,0.70}
\newcommand\Colorhref[3][color1]{\href{#2}{\small\color{#1}#3}}
\maketitle

\section{Education}
\cventry{8/15-10/19}{Ph.D, Human Genetics}{University of Utah}{Salt Lake City, UT}{}{}
\cvline{advisor:}{\small Professor Aaron Quinlan}
\cventry{8/11-6/15}{BA, Biochemistry}{Lawrence University}{Appleton, WI}{}{}
\cvline{advisor:}{\small Professor Brian Piasecki}
\cvline{honors:}{\small \emph{summa cum laude}}


\section{Experience}

\cventry{10/22-pres.}{Staff Research Scientist}{Quinlan Lab}{Univ. of Utah}{Dept. of Human Genetics}{
\begin{itemize}
    \item Developing new methods to analyze germline mutation and genome evolution 
\end{itemize}
}

\cventry{5/21-10/22.}{Senior Data Scientist}{Recursion Pharmaceuticals}{}{}{
\begin{itemize}
\item Developed new computational methods to analyze data from massive cellular imaging experiments
\item Collaborated with product managers and scientists to disseminate results to wide audiences
\item Led regular sprint planning sessions and delegating project goals to teams of ICs
\end{itemize}
}

\cventry{3/20-5/21}{Postdoctoral Fellow}{Harris Lab}{Univ. of Washington}{Dept. of Genome Sciences}{
\begin{itemize}
\item Discovered alleles that influence the germline mutation rate using whole-genome sequencing data
\item Analyzed single-cell sequencing data to characterize the mutational landscape of spermatogenesis
\item Developed reproducible workflows for processing large DNA sequencing datasets
\end{itemize}
}

\cventry{4/16-3/20}{Graduate Research Assistant}{Quinlan Lab}{Univ. of Utah}{Dept. of Human Genetics}{
\begin{itemize}
\item Analyzed whole-genome sequencing data from large multi-generational families to identify
post-zygotic mosaicism and variability in human germline mutation rates
\item Used the Oxford Nanopore Technologies platform to sequence DNA virus genomes under strong selective pressure during experimental evolution
\end{itemize}
}

\cventry{8/12-6/15}{Undergraduate Research Assistant}{Piasecki Lab}{Lawrence University}{}{
\begin{itemize}
\item Cloned and visualized the expression of genes involved in sensory
cilia structure and function
\item Constructed an automated tracking instrument to identify behavioral phenotypes
in \emph{C. elegans}
\end{itemize}
}

\section{Preprints}

\cvline{2022}{Ashbrook DG, \textbf{Sasani TA}, Maksimov M, Gunturkun MH, Ma N, Villani F, Ren Y, Rothschild D, Chen H, Lu L, Colonna V, Dumont B, Harris K, Gymrek M, Pritchard JK, Palmer AA, Williams RW. Private and sub-family specific mutations of founder haplotypes in the BXD family reveal phenotypic consequences relevant to health and disease. \Colorhref{https://www.biorxiv.org/content/10.1101/2022.04.21.489063v1.abstract}{\emph{bioRxiv.}}}

\section{Peer-reviewed manuscripts \Colorhref{https://scholar.google.com/citations?user=T_Tq0UYAAAAJ&hl=en}{(also see \underline{Google Scholar})}}

\cvline{2022}{Fixsen SM, Cone KR, Goldstein SA, \textbf{Sasani TA}, Quinlan AR, Rothenburg S, Elde NC. Poxviruses capture host genes by LINE-1 retrotransposition. \Colorhref{https://elifesciences.org/articles/63332}{\emph{eLife}}.}

\cvline{2022}{\textbf{Sasani TA}, Ashbrook DG, Beichman AC, Lu L, Palmer AA, Williams RW, Pritchard JK, Harris K. A natural mutator allele shapes mutation spectrum variation in mice. \Colorhref{https://www.nature.com/articles/s41586-022-04701-5}{\emph{Nature}}. \Colorhref{https://github.com/tomsasani/bxd_mutator_manuscript}{\textbf{Code.}}}

\cvline{2021}{Belyeu JR*, \textbf{Sasani TA}*, Pedersen BS, Quinlan AR. Unfazed: parent-of-origin detection for large and small de novo variants. \Colorhref{https://doi.org/10.1093/bioinformatics/btab454}{\emph{Bioinformatics}}.}

\cvline{2020}{Wallace AD, \textbf{Sasani TA}, Swanier J, Gates B, Greenland J, Pedersen BS, Varley KT, Quinlan AR. CaBagE: a Cas9-based Background Elimination strategy for targeted, long-read DNA sequencing. \Colorhref{https://journals.plos.org/plosone/article?id=10.1371/journal.pone.0241253}{\emph{PLoS One}}.}

\cvline{2020}{Cawthon RM, Meeks HD*, \textbf{Sasani TA}*, Smith KR, Kerber RA, O'Brien E, Baird L, Dixon MM, Peiffer AP, Leppert MF, Quinlan AR, Jorde LB. Germline mutation rates in young adults predict longevity and reproductive lifespan. \Colorhref{https://www.nature.com/articles/s41598-020-66867-0}{\emph{Scientific Reports}}.}

\cvline{2019}{\textbf{Sasani TA}, Pedersen BS, Gao Z, Baird L, Przeworski M, Quinlan AR, Jorde LB. Large, three-generation human families reveal post-zygotic mosaicism and variability in germline mutation accumulation. \Colorhref{https://elifesciences.org/articles/46922}{\emph{eLife}}. \Colorhref{https://github.com/quinlan-lab/ceph-dnm-manuscript}{\textbf{Code.}}
\Colorhref{https://www.thenakedscientists.com/articles/interviews/how-many-mutations-do-parents-pass}{Interview on the Naked Scientists podcast.}}

\cvline{2019}{Gao Z, Moorjani P, \textbf{Sasani TA}, Pedersen BS, Quinlan AR, Jorde LB, Amster G, Przeworski M.
Overlooked roles of DNA damage and maternal age in generating human germline mutations. \Colorhref{https://doi.org/10.1073/pnas.1901259116}{\emph{PNAS}}.}

\cvline{2018}{\textbf{Sasani TA*}, Cone KR*, Quinlan AR, Elde NC. Long read sequencing reveals poxvirus
evolution through rapid homogenization of gene arrays. \Colorhref{https://doi.org/10.7554/eLife.35453}{\emph{eLife}}. \Colorhref{https://github.com/tomsasani/vacv-ont-manuscript}{\textbf{Code.}}}

\cvline{2018}{Belyeu JR, Nicholas TJ, Pedersen BS, \textbf{Sasani TA}, Havrilla JM, Kravitz SN, Conway ME,
Lohman BK, Quinlan AR, Layer RM. SV-plaudit: A cloud-based framework for
manually curating thousands of structural variants. \Colorhref{https://doi.org/10.1093/gigascience/giy064}{\emph{GigaScience}}.}

\cvline{2018}{Jain M*, Koren S*, Miga KM*, Quick J*, Rand AC*, \textbf{Sasani TA*}, Tyson JR*,
Beggs AD, Dilthey AT, Fiddes IT, Malla S, Marriott H, Nieto T, O'Grady J, Olsen HE,
Pedersen BS, Rhie A, Richardson H, Quinlan AR, Snutch TP, Tee L, Paten B, Phillippy AM,
Simpson JT, Loman NJ, Loose M.
Nanopore sequencing and assembly of a human genome with ultra-long reads. \Colorhref{https://doi.org/doi:10.1038/nbt.4060}{\emph{Nature Biotechnology}}.}

\cvline{2017}{Feusier J, Witherspoon DJ, Watkins WS, Goubert C, \textbf{Sasani TA}, Jorde LB. Discovery of
rare, diagnostic Alu Yb8/9 elements in diverse human populations. \Colorhref{https://doi.org/10.1186/s13100-017-0093-0}{\emph{Mobile
DNA}}.}

\cvline{2017}{Piasecki BP, \textbf{Sasani TA}, Lessenger AT, Huth N, Farrell S. MAPK-15 is a ciliary protein
required for PKD-2 localization and male mating behavior in \emph{Caenorhabditis elegans}. \Colorhref{https://doi.org/10.1002/cm.21387}{\emph{Cytoskeleton}}.}

\cvline{*}{indicates equal contribution}

\section{Awards and Fellowships}{}
\cventry{2020-2021}{NIH T32 Postdoctoral Genome Sciences Training Grant}{University of Washington}{}{}{}
\cventry{2017-2019}{NIH T32 Predoctoral Genetics Training Grant}{University of Utah}{}{}{}
\cventry{2017 \& 2018}{Charles J. Epstein Trainee Award for Excellence in Human Genetics}{Semifinalist, American Society for Human Genetics}{}{}{}
\cventry{2018}{Lassonde Institute Student Innovator}{University of Utah}{}{}{}
\cventry{2015}{Howard and Helen Russell Award for Excellence in Biological Research}{Lawrence University}{}{}{}

\section{Invited Presentations}{}
\cvline{2021}{\emph{Mapping mutator alleles in mice}. Recent Advances in Biology Lecture Series, Lawrence University. Virtual.}
\cvline{2021}{\emph{Mapping mutator alleles in mice}. Pritchard Lab Mini-Conference, Stanford University. Virtual.}
\cvline{2021}{\emph{A wild-derived antimutator drives germline mutation spectrum differences in a genetically diverse murine family}. Przeworski Lab Meeting, Columbia University. Virtual.}
\cvline{2019}{\emph{A short tale of viral evolution told with long reads}. Society for Molecular Biology and Evolution (Annual Meeting). Manchester, UK.}
\cvline{2017}{\emph{Human immune defense mechanisms drive rapid genome evolution in vaccinia virus}. London Calling (Oxford Nanopore Technologies meeting). London, UK.}


\section{Contributed Presentations}

\cvline{2018}{\emph{Directly measuring the dynamics of the human mutation rate by sequencing large, multigenerational pedigrees}. American Society of Human Genetics, Annual Meeting. San Diego, CA, USA (Plenary Presentation).}
\cvline{2017}{\emph{Human immune defense mechanisms drive rapid genome evolution in vaccinia virus.} American Society of Human Genetics Annual Meeting. Orlando, FL, USA (Platform Presentation).}

\section{Mentorship}
\subsection{Graduate Students}
\cventry{9/20-12/20}{Candice Y.}{Rotation Project: \emph{Identifying structural variants in a recombinant inbred mouse cross}}{University of Washington}{}{}
\cventry{10/19-12/19}{Bianca A.}{Rotation Project: \emph{Patterns of recombination in large Utah pedigrees}}{University of Utah}{}{}
\cventry{1/20-3/20}{Erica H.}{Rotation Project: \emph{Regional variation in mutation rates and spectra}}{University of Utah}{}{}
\subsection{High School Students}
\cventry{7/20-9/20}{Myles F.}{Summer Project: \emph{Variation in mutation rates and spectra in a recombinant inbred mouse cross}}{University of Washington}{}{}

\section{Teaching Experience}
\subsection{Cold Spring Harbor Laboratory}
\cventry{2016 \& 2017}{Teaching Assistant}{Advanced Sequencing Technologies and Applications}{}{}{}
\subsection{University of Utah}
\cventry{2019-2020}{Guest Lecturer}{Salt Lake Learners of Biostatistics}{}{}{}
\cventry{2017}{Teaching Assistant}{Applied Computational Genomics}{}{}{}
\cventry{2016}{Teaching Assistant}{Programming for Biomedical Science}{}{}{}
\cventry{2016}{Guest Instructor}{Summer Data Science Bootcamp}{}{}{}

\section{Reviewing}
\cvline{\emph{ad hoc}}{Genome Biology, Genome Medicine, Molecular Biology and Evolution, Bioinformatics, eLife}

\section{Courses and Professional Development}
\cvline{2018}{Leena Peltonen School of Human Genomics}

\end{document}
