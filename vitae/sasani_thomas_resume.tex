%% start of file `template.tex'.
%% Copyright 2006-2010 Xavier Danaux (xdanaux@gmail.com).
%% Copyright 2010-2011 Mark Liu (markwayneliu@gmail.com).
%
% This work may be distributed and/or modified under the
% conditions of the LaTeX Project Public License version 1.3c,
% available at http://www.latex-project.org/lppl/.

\documentclass[11pt,a4paper,sans]{moderncv}

\usepackage{verbatim}

% moderncv themes
\moderncvstyle{classic}
\moderncvcolor{burgundy}

% character encoding
\usepackage[utf8]{inputenc}                   % replace by the encoding you are using

% adjust the page margins
\usepackage[scale=0.8]{geometry}
%\setlength{\hintscolumnwidth}{3cm}						% if you want to change the width of the column with the dates
%\AtBeginDocument{\setlength{\maketitlenamewidth}{6cm}}  % only for the classic theme, if you want to change the width of your name placeholder (to leave more space for your address details
%\AtBeginDocument{\recomputelengths}                     % required when changes are made to page layout lengths


% personal data
\firstname{Thomas A.}
\familyname{Sasani}
\email{thomas.a.sasani@gmail.com}                      % optional, remove the line if not wanted
\homepage{http://tomsasani.github.io}
\extrainfo{GitHub: \Colorhref{https://github.com/tomsasani}{@tomsasani}}
%\extrainfo{Twitter: \Colorhref{https://twitter.com/tomsasani}{@tomsasani}}
%\extrainfo{\url\Colorhref{https://scholar.google.com/citations?user=T_Tq0UYAAAAJ&hl=en}{Google Scholar}}                % optional, remove the line if not wanted
%\extrainfo{\url{http://markliu.me}} % optional, remove the line if not wanted

% to show numerical labels in the bibliography; only useful if you make citations in your resume
%\makeatletter
%\renewcommand*{\bibliographyitemlabel}{\@biblabel{\arabic{enumiv}}}
%\makeatother

%\nopagenumbers{}                             % uncomment to suppress automatic page numbering for CVs longer than one page
%----------------------------------------------------------------------------------
%            content
%----------------------------------------------------------------------------------
\begin{document}
\definecolor{color1}{rgb}{0.22,0.45,0.70}
\newcommand\Colorhref[3][color1]{\href{#2}{\small\color{#1}#3}}
\maketitle



\section{Education}
\cventry{2019}{Ph.D, Human Genetics}{University of Utah}{Salt Lake City, UT}{}{}
%\cvline{advisor:}{\small Professor Aaron Quinlan}
\cventry{2015}{B.A, Biochemistry}{Lawrence University}{Appleton, WI}{\emph{summa cum laude}}{}
%\cvline{advisor:}{\small Professor Brian Piasecki}

\section{Experience}

\cventry{10/22-pres.}{Staff Research Scientist}{Quinlan Lab}{Univ. of Utah}{Dept. of Human Genetics}{
\begin{itemize}
    \item Developing new methods to analyze germline mutation and genome evolution 
\end{itemize}
}

\cventry{5/21-10/22}{Senior Data Scientist}{Recursion Pharmaceuticals}{}{}{
\begin{itemize}
\item Developed new computational methods to analyze data from massive cellular imaging experiments
\item Collaborated with product managers and scientists to disseminate results to wide audiences
\item Led regular sprint planning sessions and delegating project goals to teams of ICs
\end{itemize}
}

\cventry{3/20-5/21}{Postdoctoral Fellow}{Harris Lab}{Univ. of Washington}{Dept. of Genome Sciences}{
\begin{itemize}
\item Discovered alleles that influence the germline mutation rate using whole-genome sequencing data
\item Analyzed single-cell sequencing data to characterize the mutational landscape of spermatogenesis
\item Developed reproducible workflows for processing large DNA sequencing datasets
\end{itemize}
}

\cventry{4/16-3/20}{Graduate Research Assistant}{Quinlan Lab}{Univ. of Utah}{Dept. of Human Genetics}{
\begin{itemize}
\item Analyzed whole-genome sequencing data from large multi-generational families to identify
post-zygotic mosaicism and variability in human germline mutation rates
\item Used the Oxford Nanopore Technologies platform to sequence DNA virus genomes under strong selective pressure during experimental evolution
\end{itemize}
}

\section{Skills}{}
\cventry{Programming}{Python (proficient), R (familiar), SQL (familiar)}{}{}{}{}
\cventry{Computing}{Unix, git, Sun Grid Engine, Amazon EC2}{}{}{}{}
\cventry{Data analysis}{numpy, scipy, sklearn, pandas, jupyter, unit testing frameworks (pytest)}{}{}{}{}
\cventry{Visualization}{matplotlib, ggplot2, plotly + Dash}{}{}{}{}

\section{Selected Academic Publications \Colorhref{https://scholar.google.com/citations?user=T_Tq0UYAAAAJ&hl=en}{(full list at \underline{Google Scholar})}}

\cvline{2022}{\textbf{Sasani TA}, Ashbrook DG, Beichman AC, Lu L, Palmer AA, Williams RW, Pritchard JK, Harris K. A natural mutator allele shapes mutation spectrum variation in mice. \Colorhref{https://www.nature.com/articles/s41586-022-04701-5}{\emph{Nature}}. \Colorhref{https://github.com/tomsasani/bxd_mutator_manuscript}{\textbf{Code.}}}

\cvline{2019}{\textbf{Sasani TA}, Pedersen BS, Gao Z, Baird L, Przeworski M, Quinlan AR, Jorde LB. Large, three-generation human families reveal post-zygotic mosaicism and variability in germline mutation accumulation. \Colorhref{https://elifesciences.org/articles/46922}{\emph{eLife}}. \Colorhref{https://github.com/quinlan-lab/ceph-dnm-manuscript}{\textbf{Code.}}
\Colorhref{https://www.thenakedscientists.com/articles/interviews/how-many-mutations-do-parents-pass}{Interview on the Naked Scientists podcast.}}

\section{Funding and Awards}{}
\cventry{2020 - 2021}{NIH T32 Postdoctoral Genome Sciences Training Grant}{}{}{}{}
\cventry{2017 - 2019}{NIH T32 Predoctoral Genetics Training Grant}{}{}{}{}
\cventry{2017}{Lassonde Student Innovator}{University of Utah}{}{}{}
%\cventry{2016 \& 2017}{Epstein Award Semifinalist}{American Society of Human Genetics}{}{}{}

\end{document}
